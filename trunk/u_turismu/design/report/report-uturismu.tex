\documentclass[a4paper,11pt]{report}
\usepackage[italian]{babel}
\usepackage[T1]{fontenc}
\usepackage[utf8]{inputenc}
\usepackage{lmodern}

\title{uTourism}
\author{Giuseppe La Greca, Simone Spaccarotella\\\small{\{giulagreca, spa.simone\}@gmail.com}}
\date{\small{Data di creazione: 14 gennaio 2012\\Data ultima modifica: \today}}

\begin{document}

  \maketitle
  \tableofcontents

  \begin{abstract}
    uTurismu è un sistema a supporto degli operatori turistici Calabresi. Mediante questo strumento è possibile  
    raccogliere ed organizzare tutta una serie di informazioni riguardanti i luoghi, le attività e/o gli eventi della 
    Regione. L'obiettivo primario è quello di dare agli operatori uno strumento efficace per la promozione delle proprie 
    offerte. Una vetrina sul mondo che consenta di offrire servizi sul territorio. Così facendo, si persegue quello che 
    è l'obiettivo secondario - ma non meno importante - del progetto, ovvero la promozione del territorio e lo sviluppo 
    del turismo in Calabria.

    Gli operatori turistici possono registrarsi al sistema, inserire, modificare e/o cancellare le proprie offerte 
    turistiche e venderle al cliente. È prevista un'interfaccia web mediante la quale è possibile consultare la base di 
    dati sottostante, alla ricerca di informazioni utili sui comuni calabresi, gli eventi, e quant'altro. Gli utenti che 
    vorranno registrarsi inoltre, potranno consultare i cataloghi degli operatori turistici e acquistare l'offerta più 
    vantaggiosa.
  \end{abstract}

  \chapter{Introduzione}
    Il sistema è stato progettato secondo le linee guida dell'architettura cosiddetta \textit{three-tier}. Nel nostro 
    caso è più corretto parlare di architettura \textit{three-layers}. Per la gestione della persistenza è stato 
    utilizzato Hibernate\footnote{\texttt{www.hibernate.org}}, mentre il livello di \textit{business logic} e quello di 
    \textit{presentation} sono stati gestiti mediante l'ausilio di Spring\footnote{\texttt{www.springsource.org}}.
    
\end{document}
